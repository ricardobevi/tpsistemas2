\documentclass[a4paper,11pt]{article}
\usepackage[spanish,activeacute]{babel}
\usepackage[latin1]{inputenc}
\usepackage[T1]{fontenc}
\usepackage[pdftex]{color,graphicx}
\usepackage{fancyhdr}
\usepackage{amsmath}
\usepackage{listings}
\usepackage{float}
\usepackage{color}


\floatstyle{ruled}
\newfloat{code}{hbt}{lop}
\floatname{code}{Codigo}

\lstset{
language=bash,                  % choose the language of the code
basicstyle=\footnotesize,       % the size of the fonts that are used for the code
numbers=left,                   % where to put the line-numbers
numberstyle=\footnotesize,      % the size of the fonts that are used for the line-numbers
stepnumber=1,                   % the step between two line-numbers. If it's 1 each line will be numbered
numbersep=5pt,                  % how far the line-numbers are from the code
backgroundcolor=\color{white},  % choose the background color. You must add \usepackage{color}
showspaces=false,               % show spaces adding particular underscores
showstringspaces=false,         % underline spaces within strings
showtabs=false,                 % show tabs within strings adding particular underscores
frame=tb,			% adds a frame around the code
tabsize=2,			% sets default tabsize to 2 spaces
captionpos=b,			% sets the caption-position to bottom
breaklines=true,		% sets automatic line breaking
breakatwhitespace=false,  	% sets if automatic breaks should only happen at whitespace
escapeinside={\%*}{*)},         % if you want to add a comment within your code
keywordstyle=\color{black},
stringstyle=\ttfamily,
numberstyle=\tiny,
extendedchars=true,
frameround=tftf,
inputencoding=latin1,
numberblanklines=false,
keepspaces=true
}


\newcommand{\tp}{Trabajo Pr\'actico N$^{\circ}$\numtp}
\newcommand{\?}{?`}


\pagestyle{fancy}
\renewcommand{\sectionmark}[1]{\markright{#1}{}}
\lhead{\includegraphics[width=7mm,height=7mm]{\logo} \materia}
\chead{}
\rhead{\nomtp}
\lfoot{\rightmark}
\cfoot{}
\rfoot{\thepage}
\renewcommand{\headrulewidth}{0.4pt}
\renewcommand{\footrulewidth}{0.4pt}
\renewcommand{\headheight}{24pt}


\renewcommand{\theenumii}{\arabic{enumii}}
\renewcommand{\labelenumii}{\theenumi.\theenumii}
\renewcommand{\theenumiii}{\arabic{enumiii}}
\renewcommand{\labelenumiii}{\theenumi.\theenumii.\theenumiii}


\begin{document}
%\documentclass[a4paper,11pt]{report}
%\usepackage[spanish,activeacute,es-notilde]{babel}
%\usepackage[latin1]{inputenc}
%\usepackage[T1]{fontenc}
%\usepackage[pdftex]{color,graphicx}


\newcommand{\materia}{Sistemas de Computaci\'on 2}
\newcommand{\team}{Grupo N$^{\circ}$63}
\newcommand{\comision}{3900}
\newcommand{\diacursada}{Miercoles}
\newcommand{\turno}{Noche (de 19 a 23 Hs.)}
\newcommand{\anio}{2010}
\newcommand{\fecha}{19/05/2010}
\newcommand{\numtp}{2}
\newcommand{\nomtp}{Scripts B\'asicos}
\newcommand{\docentes}{Favio Rivalta | Juan Manuel Fera | Soledad \'Alvarez}
%Poner aca la ruta al logo de la universidad.
\newcommand{\logo}{/home/ric/Documentos/Facultad/Varios/logoUnlam.png}

%\begin{document}
\thispagestyle{empty}

\begin{center}

 \includegraphics{\logo}
 % logoUnlam.png: 558x567 pixel, 200dpi, 7.09x7.20 cm, bb=0 0 201 204
 

\huge{Universidad Nacional\\de la Matanza}\\
\end{center}


\begin{center}
\rule{30mm}{.1pt}\\
\huge{\materia}\\
\huge{Trabajo Pr'actico N$^{\circ}$\numtp\\\nomtp}\\
\rule[5mm]{30mm}{.1pt}\\
\end{center}
Comisi'on: \comision.\\
Dia de Cursada: \diacursada.\\
Docentes del Curso: \docentes.\\
Turno: \turno.\\
Fecha Entrega: \fecha.\\
Curso Lectivo: \anio.



\begin{center}

\Large{Integrantes del Grupo}\\
\rule[2.5mm]{15mm}{.1pt}\\

\begin{tabular}{r|r}
Ricardo Bevilacqua & 34.304.983\\
D\'{ }Aranno Facundo Nahuel & 34.842.320
\end{tabular}



\end{center}
\newpage


%\end{document}


\setcounter{page}{1}
\pagenumbering{roman}
\tableofcontents
\newpage
\setcounter{page}{1}
\pagenumbering{arabic}

\section{Ejercicio 1}
\subsection[Punto A]{\?Cu'al es el objetivo del script?}
El objetivo es listar los usuarios del grupo ``user'' en consola.

\subsection[Punto B]{\?Cu'al es el objetivo de la primera l'inea de cualquier script escrito para los sistemas
operativos de la familia UNIX? Ejemplifique al menos 3 diferentes l'ineas y \?qu'e indicar'ian
en cada caso?}

El objetivo de la primer l'inea de los script para los sistemas de la familia UNIX indica el interprete con el cual debe ejecutar el script

\subsection[Punto C]{Comente el c'odigo del script. No es aceptable poner la descripci'on de cada uno de
los comandos, se debe realizar comentarios que ayuden a leer el script en forma m'as
r'apida y facilitar la compresi'on de la finalidad del mismo.}

\lstinputlisting[title=Punto 1]{../codigo/Punto1.sh}

\subsection[Punto D]{\?Es un shell script? \?Por qu'e?}

Si, es un shell script. Podemos distinguirlo por varias caracteristicas:

\begin{itemize}
 \item La primer linea del archivo es ``\#!/bin/bash'', lo que indica que el script debe ser leido con el interprete bash.
 \item Las lineas siguientes describen una serie de comandos que pueden ser ejecutados por un interprete.
\end{itemize}

Adem'as, en caso de tener extenci'on (lo cual no es obligatorio en UNIX) esta puede ser ``.sh'', lo que identificar'ia al archivo como un script.

\subsection[Punto E]{Proponga un ejemplo de uso.}

\begin{lstlisting}[title=Ejemplo de uso]
$./Practica1.sh
ric
\end{lstlisting}

Se puede tambien cambiar el grupo al cual hace refencia el script, pudiendo conocer los usuarios miembros de otros grupos.

\subsection[Punto F]{Supongamos que tiene que controlar la ejecuci'on paso a paso y el contenido de
las variables del script, \?qu'e comando utilizar'ia? (Tip: vea el comando set y sus
par'ametros -x, -v, -e, -u).}

Utilizar'ia el comando sed que cambia los par'ametros de ejecucion del bash. Sus opciones son las siguientes:

\begin{description}
 \item [-x] Imprime los comandos y sus argumentos antes de ser ejecutados.
 \item [-v] Imprime las lineas del shell mientras son leidas.
 \item [-e] Termina la ejecuci'on del shell si alguno de los comandos da error a la salida.
 \item [-u] Muestra un mensaje de error si el script intenta usar una variable que no esta seteada.
\end{description}

\subsection[Punto G]{Explique y ejemplifique el uso de las estructuras if test y for, mostrando distintas
formas de uso.}

\subsubsection{if test}  

Ejecuta comandos basado en una condici'on .

La estructura de la condicion if es:
\begin{lstlisting}
if test ( expresion ) then
...
fi
\end{lstlisting}

Ejemplo:

\begin{lstlisting}
#comprueba que se envien solo dos parametros y lo informa

  if test $# -eq 2
  then
          echo "se recibieron dos parametros "
  else
          echo "no se recibieron exactamente dos parametros "
  fi
\end{lstlisting}

\subsubsection{for}
Es una estructura de iteraci'on.

Ejemplo:

\begin{lstlisting}
#imprime todos los parametros recividos

  for i in $*
  do
          echo $i
  done
\end{lstlisting}



\subsection[Punto H]{Explique qu'e es un par'ametro, c'omo se pasan par'ametros a los script, c'omo se
los maneja dentro del script. \?Qu'e pasa si tengo m'as de 10 par'ametros en un script, c'omo
puedo acceder ellos?}

Un parametro en una cadena que se le envia como dato al script en su llamada, escribiendo estas a continuacion del nombre 
de este, separadas por un espacio ( o la expresion indicada en la variable \$IFS ).
Los scripts reciben los parámetros que le pasa la shell como variables nombradas de la forma \$1, \$2, ..., \$n. 
Para saber cuantos parametros hemos recibido se puede utilizar el símbolo \$\#.

Si se envian mas de diez parametros ( contando a \$0 , el nombre del script, como uno de estos ) no podremos acceder de la forma 
ya mensionada para ello deberemos utilizar el comando shift, este desplazara los parametros a la izquierda cuantas veces se 
le espeficique, por lo que en realidad no recibiremos mas de los parametros convencionales, sino que perderemos los primeros para poder 
acceder a los antes no podiamos, de esta forma si realizamos un shift, el \$1 lo perderiamos y en su lugar tendremos el 
valor que antes se ubicaba en \$2, de esta forma el parametro \$9 contendra un parametro al que antes no podiamos acceder.

Sn m'etodo muy utilizado es acceder siempre a la variable \$1 y mediante una iteracion ir realizando 
un shift para recorrer todos los parametros.

\section{Ejercicio 2}

Luego de hacer los puntos a, b y c el script modificado queda as'i:

\lstinputlisting[title=Punto 2]{../codigo/Punto2.sh}

\section{Ejercicio 3}

\lstinputlisting[title=Punto 3]{../codigo/Punto3.sh}

\section{Ejercicio 4}

\lstinputlisting[title=Punto 4]{../codigo/Punto4.sh}

\section{Ejercicio 5}

\lstinputlisting[title=Punto 5]{../codigo/Punto5.sh}

\end{document} 
