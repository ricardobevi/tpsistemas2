\documentclass[a4paper,11pt]{article}
\usepackage[spanish,activeacute]{babel}
\usepackage[latin1]{inputenc}
\usepackage[T1]{fontenc}
\usepackage[pdftex]{color,graphicx}
\usepackage{fancyhdr}
\usepackage{amsmath}
\usepackage{listings}
\usepackage{float}
\usepackage{color}


\floatstyle{ruled}
\newfloat{code}{hbt}{lop}
\floatname{code}{Codigo}

\lstset{
  language=C,                     % choose the language of the code
  basicstyle=\tiny,                 % the size of the fonts that are used for the code
  numbers=left,                   % where to put the line-numbers
  numberstyle=\footnotesize,      % the size of the fonts that are used for the line-numbers
  stepnumber=1,                   % the step between two line-numbers. If it's 1 each line will be numbered
  numbersep=5pt,                  % how far the line-numbers are from the code
  backgroundcolor=\color{white},  % choose the background color. You must add \usepackage{color}
  showspaces=false,               % show spaces adding particular underscores
  showstringspaces=false,         % underline spaces within strings
  showtabs=false,                 % show tabs within strings adding particular underscores
  frame=tb,			% adds a frame around the code
  tabsize=2,			% sets default tabsize to 2 spaces
  captionpos=b,			% sets the caption-position to bottom
  breaklines=true,		% sets automatic line breaking
  breakatwhitespace=false,  	% sets if automatic breaks should only happen at whitespace
  escapeinside={\%*}{*)},         % if you want to add a comment within your code
  stringstyle=\ttfamily,
  numberstyle=\tiny,
  extendedchars=true,
  frameround=tftf,
  inputencoding=latin1,
  numberblanklines=false,
  keepspaces=true,
  prebreak = \raisebox{0ex}[0ex][0ex]{\ensuremath{\hookleftarrow}},
  keywordstyle=\color{blue},
  commentstyle=\color[rgb]{0.5,0.5,0.5},
  stringstyle=\color{red}
}


\newcommand{\tp}{Trabajo Pr\'actico N$^{\circ}$\numtp}
\newcommand{\?}{?`}
\renewcommand{\appendixname}{Ap'endice}


\pagestyle{fancy}
\renewcommand{\sectionmark}[1]{\markright{#1}{}}
\lhead{\includegraphics[width=7mm,height=7mm]{\logo} \materia}
\chead{}
\rhead{\nomtp}
\lfoot{\rightmark}
\cfoot{}
\rfoot{\thepage}
\renewcommand{\headrulewidth}{0.4pt}
\renewcommand{\footrulewidth}{0.4pt}
\renewcommand{\headheight}{24pt}


\renewcommand{\theenumii}{\arabic{enumii}}
\renewcommand{\labelenumii}{\theenumi.\theenumii}
\renewcommand{\theenumiii}{\arabic{enumiii}}
\renewcommand{\labelenumiii}{\theenumi.\theenumii.\theenumiii}


\begin{document}
%\documentclass[a4paper,11pt]{report}
%\usepackage[spanish,activeacute,es-notilde]{babel}
%\usepackage[latin1]{inputenc}
%\usepackage[T1]{fontenc}
%\usepackage[pdftex]{color,graphicx}


\newcommand{\materia}{Sistemas de Computaci\'on 2}
\newcommand{\team}{Grupo N$^{\circ}$63}
\newcommand{\comision}{3900}
\newcommand{\diacursada}{Miercoles}
\newcommand{\turno}{Noche (de 19 a 23 Hs.)}
\newcommand{\anio}{2010}
\newcommand{\fecha}{29/09/2010}
\newcommand{\numtp}{7}
\newcommand{\nomtp}{Threads y Sockets}
\newcommand{\docentes}{Favio Rivalta | Juan Manuel Fera | Soledad \'Alvarez}
%Poner aca la ruta al logo de la universidad.
\newcommand{\logo}{/home/ric/Documentos/Facultad/Varios/logoUnlam.png}

%\begin{document}
\thispagestyle{empty}

\begin{center}

 \includegraphics{\logo}
 % logoUnlam.png: 558x567 pixel, 200dpi, 7.09x7.20 cm, bb=0 0 201 204
 

\huge{Universidad Nacional\\de la Matanza}\\
\end{center}


\begin{center}
\rule{30mm}{.1pt}\\
\huge{\materia}\\
\huge{Trabajo Pr'actico N$^{\circ}$\numtp\\\nomtp}\\
\rule[5mm]{30mm}{.1pt}\\
\end{center}
Comisi'on: \comision.\\
Dia de Cursada: \diacursada.\\
Docentes del Curso: \docentes.\\
Turno: \turno.\\
Fecha Entrega: \fecha.\\
Curso Lectivo: \anio.\\
Login Entrega: ``ricbev''.



\begin{center}

\Large{Integrantes del Grupo}\\
\rule[2.5mm]{15mm}{.1pt}\\

\begin{tabular}{r|r}
Ricardo Bevilacqua & 34.304.983\\
D\'{ }Aranno Facundo Nahuel & 34.842.320\\
Marcela A. Uslenghi & 26.920.315
\end{tabular}



\end{center}
\newpage


%\end{document}


\setcounter{page}{1}
\pagenumbering{roman}
\tableofcontents
\newpage
\setcounter{page}{1}
\pagenumbering{arabic}

\section{Ejercicio 1}

\subsection{C'odigo}
\lstinputlisting[title=ej1.c]{../cod/ej1/ej1.c}

\subsection[Punto A]{\?Alguna vez ocurre que el hilo 2 informa que el array contiene el mismo valor en todas las posiciones?}
Nunca.

\subsection[Punto B]{\?Qu'e tipo de mensajes fueron los que mayormente mostro el hilo 2? \?a qu'e se debe esto?}
El hilo 2 mostro el mensaje que indica que el array no tiene todas las posiciones iguales. Esto se debe a que el hilo 1 escribe sobre el array mientras el otro lo lee.

\subsection[Punto C]{En este ejercicio Ud. est'a utilizando un recurso compartido sin ning'un tipo de sincronizaci'on. \?Est'a bien esto? \?Puede asegurar que no exista condici'on de carrera? \?Qu'e tipos de conflictos puede traer esta modalidad de trabajo sobre el recurso compartido y por qu'e?}
No esta bien utilizar recursos compartidos sin sincronizaci'on. No es posible asegurar que no exista condici'on de carrera y cualquier hilo puede escribir o leer sobre el recurso.
En cualquier caso podemos tener datos incosistentes.

\section{Ejercicio 2}
\subsection{C'odigo}
\lstinputlisting[title=ej2.c]{../cod/ej2/ej2.c}

\subsection[Punto A]{\?Alguna vez ocurre que el hilo 2 informa que el array contiene el mismo valor en todas las posiciones?}
Siempre.

\subsection[Punto B]{\?Qu'e tipo de mensajes fueron los que mayormente mostro el hilo 2? \?a qu'e se debe esto?}
El hilo 2 mostro el mensaje que indica que el array tiene todas las posiciones iguales. Esto se debe a que se sincronizaron los hilos y se elimino la condicion de carrera.

\section{Ejercicio 3}
\subsection{C'odigo}
\subsubsection{Cliente}
\lstinputlisting[title=cliente.c]{../cod/ej3/cliente.c}

\subsubsection{Servidor}
\lstinputlisting[title=servidor.c]{../cod/ej3/servidor.c}

\section{Ejercicio 4, 5 y 6}
Este ejercicio, por su extenci'on y cantidad de archivos, decidimos no imprimirlo.


\end{document} 
