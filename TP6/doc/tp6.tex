\documentclass[a4paper,11pt]{article}
\usepackage[spanish,activeacute]{babel}
\usepackage[latin1]{inputenc}
\usepackage[T1]{fontenc}
\usepackage[pdftex]{color,graphicx}
\usepackage{fancyhdr}
\usepackage{amsmath}
\usepackage{listings}
\usepackage{float}
\usepackage{color}


\floatstyle{ruled}
\newfloat{code}{hbt}{lop}
\floatname{code}{Codigo}

\lstset{
  language=C,                     % choose the language of the code
  basicstyle=\tiny,                 % the size of the fonts that are used for the code
  numbers=left,                   % where to put the line-numbers
  numberstyle=\footnotesize,      % the size of the fonts that are used for the line-numbers
  stepnumber=1,                   % the step between two line-numbers. If it's 1 each line will be numbered
  numbersep=5pt,                  % how far the line-numbers are from the code
  backgroundcolor=\color{white},  % choose the background color. You must add \usepackage{color}
  showspaces=false,               % show spaces adding particular underscores
  showstringspaces=false,         % underline spaces within strings
  showtabs=false,                 % show tabs within strings adding particular underscores
  frame=tb,			% adds a frame around the code
  tabsize=2,			% sets default tabsize to 2 spaces
  captionpos=b,			% sets the caption-position to bottom
  breaklines=true,		% sets automatic line breaking
  breakatwhitespace=false,  	% sets if automatic breaks should only happen at whitespace
  escapeinside={\%*}{*)},         % if you want to add a comment within your code
  stringstyle=\ttfamily,
  numberstyle=\tiny,
  extendedchars=true,
  frameround=tftf,
  inputencoding=latin1,
  numberblanklines=false,
  keepspaces=true,
  prebreak = \raisebox{0ex}[0ex][0ex]{\ensuremath{\hookleftarrow}},
  keywordstyle=\color{blue},
  commentstyle=\color[rgb]{0.5,0.5,0.5},
  stringstyle=\color{red}
}


\newcommand{\tp}{Trabajo Pr\'actico N$^{\circ}$\numtp}
\newcommand{\?}{?`}
\renewcommand{\appendixname}{Ap'endice}


\pagestyle{fancy}
\renewcommand{\sectionmark}[1]{\markright{#1}{}}
\lhead{\includegraphics[width=7mm,height=7mm]{\logo} \materia}
\chead{}
\rhead{\nomtp}
\lfoot{\rightmark}
\cfoot{}
\rfoot{\thepage}
\renewcommand{\headrulewidth}{0.4pt}
\renewcommand{\footrulewidth}{0.4pt}
\renewcommand{\headheight}{24pt}


\renewcommand{\theenumii}{\arabic{enumii}}
\renewcommand{\labelenumii}{\theenumi.\theenumii}
\renewcommand{\theenumiii}{\arabic{enumiii}}
\renewcommand{\labelenumiii}{\theenumi.\theenumii.\theenumiii}


\begin{document}
%\documentclass[a4paper,11pt]{report}
%\usepackage[spanish,activeacute,es-notilde]{babel}
%\usepackage[latin1]{inputenc}
%\usepackage[T1]{fontenc}
%\usepackage[pdftex]{color,graphicx}


\newcommand{\materia}{Sistemas de Computaci\'on 2}
\newcommand{\team}{Grupo N$^{\circ}$63}
\newcommand{\comision}{3900}
\newcommand{\diacursada}{Miercoles}
\newcommand{\turno}{Noche (de 19 a 23 Hs.)}
\newcommand{\anio}{2010}
\newcommand{\fecha}{21/07/2010}
\newcommand{\numtp}{6}
\newcommand{\nomtp}{Sem\'aforos y Memoria Compartida}
\newcommand{\docentes}{Favio Rivalta | Juan Manuel Fera | Soledad \'Alvarez}
%Poner aca la ruta al logo de la universidad.
\newcommand{\logo}{/home/ric/Documentos/Facultad/Varios/logoUnlam.png}

%\begin{document}
\thispagestyle{empty}

\begin{center}

 \includegraphics{\logo}
 % logoUnlam.png: 558x567 pixel, 200dpi, 7.09x7.20 cm, bb=0 0 201 204
 

\huge{Universidad Nacional\\de la Matanza}\\
\end{center}


\begin{center}
\rule{30mm}{.1pt}\\
\huge{\materia}\\
\huge{Trabajo Pr'actico N$^{\circ}$\numtp\\\nomtp}\\
\rule[5mm]{30mm}{.1pt}\\
\end{center}
Comisi'on: \comision.\\
Dia de Cursada: \diacursada.\\
Docentes del Curso: \docentes.\\
Turno: \turno.\\
Fecha Entrega: \fecha.\\
Curso Lectivo: \anio.\\
Login Entrega: ``ricbev''.



\begin{center}

\Large{Integrantes del Grupo}\\
\rule[2.5mm]{15mm}{.1pt}\\

\begin{tabular}{r|r}
Ricardo Bevilacqua & 34.304.983\\
D\'{ }Aranno Facundo Nahuel & 34.842.320\\
Jose Ferreyra & 31.144.004\\
Marcela A. Uslenghi & 26.920.315
\end{tabular}



\end{center}
\newpage


%\end{document}


\setcounter{page}{1}
\pagenumbering{roman}
\tableofcontents
\newpage
\setcounter{page}{1}
\pagenumbering{arabic}

\section{Ejercicio 1}

\subsection{C'odigo}
\lstinputlisting[title=ej1.c]{../cod/ej1/ej1.c}

\subsection[Punto A]{\?Cu'al es el tama\~no que se reserva de memoria compartida?}
El tama\~no de MC reservado es de 80 bits (10 bytes).

\subsection[Punto B]{\?Qu'e se almacena en la memoria compartida?}
En la memoria compartida se almacenan diez letras ``a''.

\subsection[Punto C]{En una sesi'on distinta a la que est'a ejecutando el proceso, ingrese el comando ``ipcs'' y analice el resultado. \?Puede explicar qu'e est'a informando el comando?}
Al ejecutar el comando ipcs me muestra los segmentos de memoria compartida que estan en uso. El ultimo dato mostrado es el que se reservo en el proceso que se ejecuto del ejercicio 1 del 6.

\subsection[Punto D]{Si ejecuta ``ipcs'' durante los primeros 10 segundos que 'este proceso duerme, aparece cierta informaci'on, y si lo hace en los pr'oximos 10 segundos aparece un cambio. \?En qu'e consiste?}

Al ejecutar el comando ipcs durante los diez segundos que duerme el proceso, me muestra el segmento asignado para el mismo. ( 10 bytes ) luego que pasaron los diez segundos dicha memoria asignada no vuelve a aparecer dentro del listado.

\subsection[Punto E]{El primer valor de la salida de ``ipcs'' es ``key''. \?Qu'e relaci'on tiene ese valor que apareci'o con respecto a lo definido en la constante LLAVE?}

Al ejecutar el comando ipcs, el valor 234 asignado como LLAVE, esta representada en Hexadecimal en la columna correspondiente ``key'' ( 0x000000ea).

\subsection[Punto F]{Coloque comentarios en cada bloque funcional del c'odigo.}

Ver c'odigo arriba.

\section{Ejercicio 2}
\subsection{C'odigo}
\lstinputlisting[title=ej2Abecedario.c]{../cod/ej2/ej2Abecedario.c}

\section{Ejercicio 3}
\subsection{C'odigo}
\subsubsection{A}
\lstinputlisting[title=ej3A.c]{../cod/ej3/ej3A.c}

\subsubsection{B}
\lstinputlisting[title=ej3B.c]{../cod/ej3/ej3B.c}

\section{Ejercicio 4}
\subsection{C'odigo}
\lstinputlisting[title=ej4.c]{../cod/ej4/ej4.c}

\subsection{semaforo.h (del ejercicio 4)}
\lstinputlisting[title=semaforo.h]{../cod/ej4/semaforo.h}

\appendix
\section{semaforo.h}
Esta es la versi'on de semaforo.h usada en los puntos 2 y 3 del presente trabajo.
\lstinputlisting[title=semaforo.h]{../cod/ej2/semaforo.h}

\end{document} 
