\documentclass[a4paper,11pt]{article}
\usepackage[spanish,activeacute]{babel}
\usepackage[latin1]{inputenc}
\usepackage[T1]{fontenc}
\usepackage[pdftex]{color,graphicx}
\usepackage{fancyhdr}
\usepackage{amsmath}
\usepackage{listings}
\usepackage{float}
\usepackage{color}


\floatstyle{ruled}
\newfloat{code}{hbt}{lop}
\floatname{code}{Codigo}

\lstset{
language=bash,                  % choose the language of the code
basicstyle=\footnotesize,       % the size of the fonts that are used for the code
numbers=left,                   % where to put the line-numbers
numberstyle=\footnotesize,      % the size of the fonts that are used for the line-numbers
stepnumber=1,                   % the step between two line-numbers. If it's 1 each line will be numbered
numbersep=5pt,                  % how far the line-numbers are from the code
backgroundcolor=\color{white},  % choose the background color. You must add \usepackage{color}
showspaces=false,               % show spaces adding particular underscores
showstringspaces=false,         % underline spaces within strings
showtabs=false,                 % show tabs within strings adding particular underscores
frame=tb,			% adds a frame around the code
tabsize=2,			% sets default tabsize to 2 spaces
captionpos=b,			% sets the caption-position to bottom
breaklines=true,		% sets automatic line breaking
breakatwhitespace=false,  	% sets if automatic breaks should only happen at whitespace
escapeinside={\%*}{*)},         % if you want to add a comment within your code
keywordstyle=\color{black},
stringstyle=\ttfamily,
numberstyle=\tiny,
extendedchars=true,
frameround=tftf,
inputencoding=latin1,
numberblanklines=false,
keepspaces=true
}


\newcommand{\tp}{Trabajo Pr\'actico N$^{\circ}$\numtp}
\newcommand{\?}{?`}


\pagestyle{fancy}
\renewcommand{\sectionmark}[1]{\markright{#1}{}}
\lhead{\includegraphics[width=7mm,height=7mm]{\logo} \materia}
\chead{}
\rhead{\nomtp}
\lfoot{\rightmark}
\cfoot{}
\rfoot{\thepage}
\renewcommand{\headrulewidth}{0.4pt}
\renewcommand{\footrulewidth}{0.4pt}
\renewcommand{\headheight}{24pt}


\renewcommand{\theenumii}{\arabic{enumii}}
\renewcommand{\labelenumii}{\theenumi.\theenumii}
\renewcommand{\theenumiii}{\arabic{enumiii}}
\renewcommand{\labelenumiii}{\theenumi.\theenumii.\theenumiii}


\begin{document}
%\documentclass[a4paper,11pt]{report}
%\usepackage[spanish,activeacute,es-notilde]{babel}
%\usepackage[latin1]{inputenc}
%\usepackage[T1]{fontenc}
%\usepackage[pdftex]{color,graphicx}


\newcommand{\materia}{Sistemas de Computaci\'on 2}
\newcommand{\team}{Grupo N$^{\circ}$63}
\newcommand{\comision}{3900}
\newcommand{\diacursada}{Miercoles}
\newcommand{\turno}{Noche (de 19 a 23 Hs.)}
\newcommand{\anio}{2010}
\newcommand{\fecha}{09/06/2010}
\newcommand{\numtp}{3}
\newcommand{\nomtp}{Programaci\'on de Scripts con AWK}
\newcommand{\docentes}{Favio Rivalta | Juan Manuel Fera | Soledad \'Alvarez}
%Poner aca la ruta al logo de la universidad.
\newcommand{\logo}{/home/ric/Documentos/Facultad/Varios/logoUnlam.png}

%\begin{document}
\thispagestyle{empty}

\begin{center}

 \includegraphics{\logo}
 % logoUnlam.png: 558x567 pixel, 200dpi, 7.09x7.20 cm, bb=0 0 201 204
 

\huge{Universidad Nacional\\de la Matanza}\\
\end{center}


\begin{center}
\rule{30mm}{.1pt}\\
\huge{\materia}\\
\huge{Trabajo Pr'actico N$^{\circ}$\numtp\\\nomtp}\\
\rule[5mm]{30mm}{.1pt}\\
\end{center}
Comisi'on: \comision.\\
Dia de Cursada: \diacursada.\\
Docentes del Curso: \docentes.\\
Turno: \turno.\\
Fecha Entrega: \fecha.\\
Curso Lectivo: \anio.



\begin{center}

\Large{Integrantes del Grupo}\\
\rule[2.5mm]{15mm}{.1pt}\\

\begin{tabular}{r|r}
Ricardo Bevilacqua & 34.304.983\\
D\'{ }Aranno Facundo Nahuel & 34.842.320\\
Moure Pablo & 32.031.459\\
Marcela A. Uslenghi & 26.920.315
\end{tabular}



\end{center}
\newpage


%\end{document}


\setcounter{page}{1}
\pagenumbering{roman}
\tableofcontents
\newpage
\setcounter{page}{1}
\pagenumbering{arabic}

\section{Ejercicio 1}
\subsection{Script}
\lstinputlisting[title=Script Punto 1]{../codigo/ejercicio1[chmod]/Punto1.sh}
\subsection{AWK}
\lstinputlisting[title=Script Punto 1]{../codigo/ejercicio1[chmod]/p1Formato.awk}

\section{Ejercicio 2}
\subsection{Script}
\lstinputlisting[title=Script Punto 2]{../codigo/Ejercicio2-password/generador-de-pass.sh}

\section{Ejercicio 3}
\subsection{Script}
\lstinputlisting[title=Script Punto 3]{../codigo/ejercicio3/ejercicio3.sh}

\end{document} 
